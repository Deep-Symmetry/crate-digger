\documentclass[11pt]{article}

\usepackage[utf8]{inputenc}
\usepackage[english]{babel}
\usepackage{courier}
\usepackage{graphicx}
\usepackage[bottom]{footmisc}
\usepackage{bytefield}
\usepackage{hyperref}
\usepackage{bookmark}
\usepackage{color}
\usepackage{alltt}
\usepackage{tabu, booktabs}
\usepackage{longtable}

\newcommand{\rmst}{{\it\small rmst}}

\title{Rekordbox Export Structure Analysis}
\author{James Elliott\\Deep Symmetry, LLC}

\begin{document}

\maketitle

\abstract{The files written to external media by rekordbox for use in
  player hardware contain a wealth of information that can be used in
  place of queries to the remotedb server on the players, which is
  important because they can be obtained from the players' NFS
  servers, even if there are four players in use sharing the same
  media. Under those circumstances, remotedb queries are impossible.
  This article documents what has been learned so far about the
  files, and how to interpret them.}

\pagestyle{headings}

%% Define macros used to draw more complex message byte fields with
%% labeled headers and color-related sections.

\newcommand\hexhead{
  \bitbox[]{1}{\tiny\bfseries 0}
  \bitbox[]{1}{\tiny\bfseries 1}
  \bitbox[]{1}{\tiny\bfseries 2}
  \bitbox[]{1}{\tiny\bfseries 3}
  \bitbox[]{1}{\tiny\bfseries 4}
  \bitbox[]{1}{\tiny\bfseries 5}
  \bitbox[]{1}{\tiny\bfseries 6}
  \bitbox[]{1}{\tiny\bfseries 7}
  \bitbox[]{1}{\tiny\bfseries 8}
  \bitbox[]{1}{\tiny\bfseries 9}
  \bitbox[]{1}{\tiny\bfseries a}
  \bitbox[]{1}{\tiny\bfseries b}
  \bitbox[]{1}{\tiny\bfseries c}
  \bitbox[]{1}{\tiny\bfseries d}
  \bitbox[]{1}{\tiny\bfseries e}
  \bitbox[]{1}{\tiny\bfseries f}
}

\newcommand{\baselinealign}[1]{%
  \centering
  \strut#1%
}

\newcommand{\colorbitbox}[4][rlbt]{%
  \sbox0{\bitbox[#1]{#3}{#4}}%
 \makebox[0pt][l]{\textcolor{#2}{\rule[-\dp0]{\wd0}{\ht0}}}%
 \bitbox[#1]{#3}{#4}%
}

\definecolor{lightgreen}{rgb}{0.64,1,0.71}
\definecolor{yellow}{rgb}{1,1,0.71}
\definecolor{lightred}{rgb}{1,0.7,0.71}
\definecolor{lightcyan}{rgb}{0.84,1,1}
\definecolor{lightpurple}{rgb}{1,0.71,1}

\tableofcontents

\newpage

\section{Database Exports}

The starting point for finding track metadata from a player is the
database export file, which can be found within rekordbox media at the
path {\tt PIONEER/rekordbox/export.pdb} (if you are using the Crate
Digger {\tt FileFetcher} to request this file, use that path as the
{\tt filePath} argument, and use a {\tt mountPath} value of {\tt /B/}
if you want to read it from the SD slot, or {\tt /C/} to obtain it
from the USB slot).

The file consists of a series of fixed size pages. The first page
contains a file header which defines the page size and the locations
of database tables of different types, by the index of their first
page. The rest of the pages consist of the data pages for all of the
tables identified in the header. Each page contains rows of a single
table, and links to the next page containing that type of rows.

\subsection{File Header}

\section{Crate Digger}

You can find a Java library that can parse the structures described in
this research, and that can retrieve them from players' NFS servers,
at: \url{https://github.com/deep-symmetry/crate-digger}

The project also contains Kaitai Struct specifications for the file
structures, which were used to automatically generate the Java classes
which parse them, and which can be used to generate equivalent code
for a variety of other programming languages.

There are also ONC RPC specification files that were similarly used to
generate the Java classes used to send and receive messages to the
NFSv2 servers in the players to request these files, and which can
likely be used to generate structures for other languages as well.

\begin{appendix}

  \addcontentsline{toc}{section}{\listfigurename}
  \listoffigures

  \addcontentsline{toc}{section}{\listtablename}
  \listoftables

  \begin{center}
    \begin{samepage}
      \includegraphics[width=4cm]{assets/DS-Logo-bw-4k}

      \vspace{0.25cm}
      \url{http://deepsymmetry.org}
    \end{samepage}
  \end{center}

\end{appendix}

\end{document}
